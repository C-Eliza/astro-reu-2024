\documentclass[aspectratio=169,compress]{beamer}
\usetheme{Szeged}
\usecolortheme{beaver}
\graphicspath{ {./figures} }

\title{Billowing Hydrogen}
\subtitle{Simulating Turbulence in HII Regions}
\author[Canales, Wenger]{Eliza Canales \& Trey Wenger}
\institute[UW-Madison \& NRAO]{University of Wisconsin -- Madison \& National Radio Astronomy Observatory}
\date{July 31st, 2024}
\logo{
  \includegraphics[height=1.25cm]{uwlogo.png}
  \hspace{2mm}
  \includegraphics[height=1.25cm]{nraologo.png}
}

\setbeamercolor{def}{fg=white,bg=darkred}

\setbeamertemplate{headline}{
  \leavevmode%
  \hbox{\begin{beamercolorbox}[wd=\paperwidth,ht=4.5ex,dp=1.125ex,leftskip=.3cm,rightskip=.3cm]{def}
  \end{beamercolorbox}}
  \vskip0pt%
}
\setbeamertemplate{footline}{
  \leavevmode%
  \hbox{\begin{beamercolorbox}[wd=\paperwidth,ht=4.5ex,dp=1.125ex,leftskip=.3cm,rightskip=.3cm]{def}
  \end{beamercolorbox}}
  \vskip0pt%
}
\setbeamertemplate{navigation symbols}{}

\begin{document}

\begin{frame}[noframenumbering]
  \titlepage
\end{frame}

\setbeamertemplate{footline}
{%
  \leavevmode%
  \hbox{\begin{beamercolorbox}[wd=.4\paperwidth,ht=4.5ex,dp=1.125ex,leftskip=.3cm,rightskip=.3cm]{def}%
  \insertshorttitle
  \end{beamercolorbox}%
  \begin{beamercolorbox}[wd=.4\paperwidth,ht=4.5ex,dp=1.125ex,leftskip=.3cm,rightskip=.3cm plus1fil]{def}%
    \usebeamerfont{author in head/foot}\insertshortauthor
  \end{beamercolorbox}}%
  \begin{beamercolorbox}[wd=.2\paperwidth,ht=4.5ex,dp=1.125ex,leftskip=.3cm,rightskip=.3cm]{def}%
  \insertshortinstitute\hfill\insertframenumber
  \end{beamercolorbox}%
  \vskip0pt%
}

\begin{frame}{Outline}
  \tableofcontents
\end{frame}

\section{Introduction}
\subsection{HII Regions}
\begin{frame}
  \frametitle{HII Regions}
  \begin{itemize}
    \item What is HII?
    \item Physical Traits
      \begin{itemize}
        \item Powered by hot stars
        \item Can range from AU to parsecs
        \item A type of nebulae
      \end{itemize}
  \end{itemize}
\end{frame}

\subsection{Emissions}
\begin{frame}
  \frametitle{Emissions}
  \begin{itemize}
    \item A way we observe HII regions
    \item Free-free continuum
    \item Radio recombination lines (RRLs)
  \end{itemize}
\end{frame}

\subsection{Radio Imaging}
\begin{frame}
  \frametitle{Radio Imaging}
  \begin{itemize}
    \item Multiple frequencies
    \item Doppler shift
    \item Mapping velocity
  \end{itemize}
\end{frame}

\subsection{Turbulence}
\begin{frame}
  \frametitle{Turbulence}
  \begin{itemize}
    \item Using turbustat
      \begin{itemize}
        \item Python package
        \item Cubes of density and velocity
      \end{itemize}
    \item Special considerations made for RRL case
  \end{itemize}
\end{frame}

\section{Motivations}
\begin{frame}
  \frametitle{Motivations}
  \begin{itemize}
    \item Previous work had shown what looked like rotation
    \item Later observations show a more complex story
    \item Can turbulence explain this behavior?
  \end{itemize}
\end{frame}

\section{Project Goals}
\begin{frame}
  \frametitle{Project Goals}
  \begin{itemize}
    \item Simulate turbulence in HII regions
      \begin{itemize}
        \item Calculating emission for each voxel
        \item Overlaying densities and velocities
      \end{itemize}
    \item Test different turbulence parameters
      \begin{itemize}
        \item Comparing to reality
        \item Use statistical models to compare
      \end{itemize}
  \end{itemize}
\end{frame}

\section{Results}
\begin{frame}
  \frametitle{Results}
  Images here
\end{frame}

\begin{frame}
  \frametitle{Results}
  \begin{itemize}
    \item Similarity to reality
      \begin{itemize}
        \item Turbulence looking like angular momentum
        \item Similar velocity scales
      \end{itemize}
  \end{itemize}
\end{frame}

\section{Future Work}
\begin{frame}
  \frametitle{Future Work}
  \begin{itemize}
    \item Comparing with more radio data
    \item Refining simulation
    \item Testing under various conditions
  \end{itemize}
\end{frame}

\section{Conclusion}
\begin{frame}
  \frametitle{Conclusion}
  \begin{itemize}
    \item Turbulence can explain what we see!
  \end{itemize}
\end{frame}

\end{document}
