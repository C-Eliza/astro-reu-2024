\documentclass[aspectratio=169,compress]{beamer}
\usepackage{bookmark}
\usepackage{hyperref}
\usetheme{Szeged}
\usecolortheme{seahorse}
\graphicspath{ {./figures} }

\definecolor{nmtblue}{HTML}{0C2753}
\definecolor{nmtorange}{HTML}{78290C}
\definecolor{nmtmid}{HTML}{6275AB}
\definecolor{nmtlight}{HTML}{BCCEFF}

\newcommand\teeny{\fontsize{3pt}{3.6pt}\selectfont}

\title[\color{nmtlight} Billowing Hydrogen]{\color{nmtblue} Billowing Hydrogen}
\subtitle{\color{nmtblue} Simulating Turbulence in HII Regions}
\author[Canales, Wenger, Svoboda]{Eliza Canales, Trey Wenger, \& Brian Svoboda}
\institute[NMT, CSUChico, \& NRAO]{New Mexico Tech, California State University Chico, \& National Radio Astronomy Observatory}
\date{November 20th, 2025}
\logo{
  \includegraphics[height=1.25cm]{nmt.png}
  \hspace{2mm}
  \includegraphics[trim = 75 75 75 75,height=1.25cm]{csuchico.png}
  \hspace{2mm}
  \includegraphics[height=1.25cm]{nraologo.png}
}

\setbeamercolor{def}{fg=nmtlight,bg=nmtblue}

\setbeamertemplate{headline}{
  \leavevmode%
  \hbox{\begin{beamercolorbox}[wd=\paperwidth,ht=4.5ex,dp=1.125ex,leftskip=.3cm,rightskip=.3cm]{def}
  \end{beamercolorbox}}
  \vskip0pt%
}
\setbeamertemplate{footline}{
  \leavevmode%
  \hbox{\begin{beamercolorbox}[wd=\paperwidth,ht=4.5ex,dp=1.125ex,leftskip=.3cm,rightskip=.3cm]{def}
  \end{beamercolorbox}}
  \vskip0pt%
}
\setbeamertemplate{navigation symbols}{}

\hypersetup{
    colorlinks=true,
    urlcolor=cyan,
    }
\urlstyle{same}

\begin{document}

\begin{frame}[noframenumbering]
  \titlepage
\end{frame}

\setbeamertemplate{footline}
{%
  \leavevmode%
  \hbox{\begin{beamercolorbox}[wd=.4\paperwidth,ht=4.5ex,dp=1.125ex,leftskip=.3cm,rightskip=.3cm]{def}%
  \insertshorttitle
  \end{beamercolorbox}%
  \begin{beamercolorbox}[wd=.4\paperwidth,ht=4.5ex,dp=1.125ex,leftskip=.3cm,rightskip=.3cm plus1fil]{def}%
    \usebeamerfont{author in head/foot}\insertshortauthor
  \end{beamercolorbox}}%
  \begin{beamercolorbox}[wd=.2\paperwidth,ht=4.5ex,dp=1.125ex,leftskip=.3cm,rightskip=.3cm]{def}%
  \hfill\insertframenumber
  \end{beamercolorbox}%
  \vskip0pt%
}

\setbeamertemplate{logo}{}
\begin{frame}
  \frametitle{HII Regions}
  \begin{columns}
    \begin{column}{0.5\textwidth}%%
      \begin{itemize}
        \item What is an HII Region?
        \item Physical Traits
          \begin{itemize}
            \item Powered by hot stars
            \item Can range from AU to parsecs
            \item A type of nebulae
          \end{itemize}
      \end{itemize}
    \end{column}
    \begin{column}{0.5\textwidth}%%
      \teeny
      \includegraphics[width=0.6\textwidth]{nebula.jpg}
      {\teeny\\ Image of an HII region, the Trifid Nebula.}
      {\teeny\\ Nebula image: M20 | Trifid Nebula HII Region in Sagittarius 6° from Kaus Borealis (top of the teapot)} 
      {\teeny\\ taken by R Jay GaBany}
      
    \end{column}
  \end{columns}
\end{frame}

\begin{frame}
  \frametitle{Observing HII Regions}
		\centering
		\includegraphics[width=0.45\textwidth]{nebula.jpg}
\end{frame}

\begin{frame}
  \frametitle{Observing HII Regions}
		\centering
		\includegraphics[width=0.45\textwidth]{nebulathicc.jpg}
\end{frame}

\begin{frame}
  \frametitle{Radio Recombination Lines}
		\centering
      \includegraphics[width=0.45\textwidth]{rrl.jpg}
      {\teeny\\ Radio recombination lines: \url{https://astronoo.com/images/lumiere/absorption-et-emission.jpg}}
\end{frame}

\begin{frame}
  \frametitle{Radio Recombination Lines}
		\begin{columns}
				\begin{column}{0.5\linewidth}
		\centering
      \includegraphics[width=0.90\textwidth]{rrl.jpg}
				\end{column}
				\begin{column}{0.5\linewidth}
				\begin{itemize}
						\item From n 3-2: 656 nm / 457 THz (Red)
						\item From n 101-100: $\sim$5 cm / 6.5 GHz (Radio)
				\end{itemize}
				\end{column}
		\end{columns}
\end{frame}

\begin{frame}
  \frametitle{Radio Recombination Lines}
		\begin{columns}
				\begin{column}{0.5\linewidth}
		\centering
      \includegraphics[width=0.90\textwidth]{rrl.jpg}
				\end{column}
				\begin{column}{0.5\linewidth}
				\begin{itemize}
						\item Not too bright
						\item Doesn't get reabsorbed
				\end{itemize}
				\end{column}
		\end{columns}
\end{frame}

\begin{frame}
  \frametitle{Radio Imaging}
  \begin{columns}
    \begin{column}{0.5\textwidth}
      \begin{itemize}
        \item Multiple frequencies
        \item Doppler shift
        \item Velocity compared to Local Standard of Rest (VLSR)
		\item Velocity line width ($\Delta V$)
      \end{itemize}
    \end{column}
    \begin{column}{0.5\textwidth}
			\centering
      \includegraphics[width=0.8\textwidth]{rrlspectrum.png}
			{\teeny\\ Typical radio recombination line spectrum. The Galactic H II region OA 184. Available \href{https://www.researchgate.net/figure/Radio-recombination-line-emission-from-OA-184-This-is-a-composite-spectrum-of-seven\_fig1\_44090151}{here.}}
    \end{column}
  \end{columns}
\end{frame}

\begin{frame}
  \frametitle{Radio Imaging}
  \begin{columns}
    \begin{column}{0.5\textwidth}
      \begin{itemize}
        \item Multiple frequencies
        \item Doppler shift
        \item Velocity compared to Local Standard of Rest (VLSR)
        \item Velocity line width ($\Delta V$)
      \end{itemize}
    \end{column}
    \begin{column}{0.5\textwidth}
      \includegraphics[width=0.8\textwidth]{rrlspectrumhelper.png}
    \end{column}
  \end{columns}
\end{frame}

\begin{frame}
  \frametitle{Radio Imaging}
  \begin{columns}
    \begin{column}{0.4\textwidth}
      \begin{itemize}
        \item Multiple frequencies
        \item Doppler shift
        \item Velocity compared to Local Standard of Rest (VLSR)
        \item Velocity line width ($\Delta V$)
      \end{itemize}
    \end{column}
    \begin{column}{0.6\textwidth}
      \centering
      \includegraphics[width=0.45\textwidth]{ch136.all.I.channel.clean.pbcor.imsmooth.image.linevlsr.png}
      \includegraphics[width=0.45\textwidth]{ch136.all.I.channel.clean.pbcor.imsmooth.image.linefwhm.png}
      {\teeny\\ First and second moment maps of the HII region RCW87.}
    \end{column}
  \end{columns}
\end{frame}

\begin{frame}
  \frametitle{Emission Line-Broadening}
  \begin{columns}
    \begin{column}{0.3\linewidth}
      \begin{itemize}
        \item Thermal motion
        \item Bulk motion
          \begin{itemize}
            \item Outflow
            \item Expansion
            \item Rotation
          \end{itemize}
        \item Turbulence
      \end{itemize}
    \end{column}
    \begin{column}{0.7\linewidth}
      \centering
      \includegraphics[height=1in]{figures/fire.png}\\
      \includegraphics[height=1in]{figures/cycle.png}
      \includegraphics[height=1in]{figures/wind.png}
			{\teeny\\ Fire source image \href{https://www.vecteezy.com/png/19787026-fire-icon-on-transparent-background}{here}, cycle image source \href{https://www.vecteezy.com/png/18723264-roundabout-directional-arrow-sign-on-transparent-background}{here}, and wind image source \href{https://www.vecteezy.com/png/22183351-hand-drawn-doodle-vaporize-icon}{here}}
    \end{column}
  \end{columns}
\end{frame}

\begin{frame}
  \frametitle{Motivations}
  \begin{itemize}
    \item Previous work had shown what looked like rotation
    \item Later observations show a more complex story
    \item Can turbulence alone explain this behavior?
  \end{itemize}
  \centering
  \vspace{3mm}
  \includegraphics[width=0.3\linewidth]{ch136.all.I.channel.clean.pbcor.imsmooth.image.linevlsr.png}
  \includegraphics[width=0.3\linewidth]{g320.channel.uvtaper.16stack.image.imsmooth.20arcsec.pbcor.vlsr.png}
  {\teeny\\ Showing the how the same object can appear different based on the beam width.}

\end{frame}

\begin{frame}
  \frametitle{Turbulence}
      \centering
      \includegraphics[width=0.8\linewidth]{figures/turbulence.jpg}\\
			{\teeny \href{https://www.advancedsciencenews.com/wp-content/uploads/2023/07/swirl-g52ac5d4ac_1280.jpg}{Image attribution}}
\end{frame}

\begin{frame}
  \frametitle{Project Goals}
  \begin{itemize}
    \item Simulate turbulence in HII regions
    \item Compare to reality qualitatively
    \item Characterize HII regions by their observables
  \end{itemize}
\end{frame}

\begin{frame}
  \frametitle{Results}
  \centering
  \includegraphics[width=0.6\linewidth]{combined_obs.png}
\end{frame}

\begin{frame}
  \frametitle{Results}
  \begin{columns}
    \begin{column}{0.5\linewidth}
      \begin{itemize}
        \item Similarity to reality
          \begin{itemize}
            \item Turbulence looking like angular momentum
            \item Similar velocity scales
          \end{itemize}
      \end{itemize}
    \end{column}
		  \begin{column}{0.5\linewidth}
		\centering
      \includegraphics[width=0.6\linewidth]{g320.channel.uvtaper.16stack.image.imsmooth.20arcsec.pbcor.vlsr.png}\\
      \includegraphics[width=0.6\linewidth]{run_s15_md5.0rrl_100.0_M1.png}
\end{column}
  \end{columns}
\end{frame}

\begin{frame}
  \frametitle{Correlation}
  \begin{columns}
    \begin{column}{0.5\linewidth}
			\centering
      \includegraphics[width=0.8\linewidth]{power1law.png}
    \end{column}
    \begin{column}{0.5\linewidth}
      \begin{itemize}
        \item We can quanitify how related a pixel is to nearby pixels
        \item The more extreme the slope, the less related to far pixels it is
      \end{itemize}
    \end{column}
  \end{columns}
\end{frame}

\begin{frame}
  \frametitle{Correlation}
  \begin{columns}
    \begin{column}{0.5\linewidth}
			\centering
      \includegraphics[width=0.85\linewidth]{aveg_1power.png}
    \end{column}
    \begin{column}{0.5\linewidth}
			\centering
      \includegraphics[width=0.85\linewidth]{aveg_2power.png}
    \end{column}
  \end{columns}
\end{frame}

\begin{frame}
  \frametitle{At Last, a Sanity Check}
			\centering
      \includegraphics[width=0.85\linewidth]{aveg_moment2.png}
\end{frame}

\setbeamertemplate{footline}{
  \leavevmode%
  \hbox{\begin{beamercolorbox}[wd=\paperwidth,ht=4.5ex,dp=1.125ex,leftskip=.3cm,rightskip=.3cm]{def}
  \end{beamercolorbox}}
  \vskip0pt%
}

\begin{frame}[noframenumbering]
  \frametitle{Pipeline}
  \begin{itemize}
    \item Generate and truncate turbustat data
      \begin{itemize}
        \item Creates cubes to represent density and velocity in 3d space
      \end{itemize}
    \item Calculate emission measure for each physical "voxel" of HII region
    \item Calculate RRL strength for each pixel
      \begin{itemize}
        \item Gaussian treating velocity cube as line centers
        \item Add free-free emission afterwards
      \end{itemize}
  \end{itemize}
\end{frame}

\begin{frame}[noframenumbering]
  \frametitle{Resolution Dependance}
  \centering
  \includegraphics[width=0.6\linewidth]{figures/vel_histogram.png}
  {\teeny\\ Demonstrating that the resolution dependance is negligible past 300 pixels.}
\end{frame}

\end{document}
