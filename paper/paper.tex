% Eliza Canales
\documentclass{article}

\begin{document}
\section{Introduction}

The motions and movements of HII regions are largely not understood. Spectral data taken in the past has suggested that HII regions may have a simple spin about an axis, but followup observations seem to shoo away this theory. It may be suggested that there is a turbulance that drives the discrepancies between the more simple model and the observations. To do this, we observe the HII region [put the important HII region here or data set]. Notably this/these HII regions have an unusual shape that [is important because why?]. Bipolar HII regions such as these are HII regions that are shaped so they have two lobes, typically thought to be shaped by outflows from energetic young stars. The goal of this work is to analyze the link between the kinematics of the hotter HII region and the motion of the surrounding ISM. This will be done by comparing spectral data of radio recombination lines from the HII region and the infrared spectral data of surrounding neutral gas. 
\end{document}
