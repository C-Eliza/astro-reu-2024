% Eliza Canales
\documentclass{article}
\def\code#1{\texttt{#1}}

\begin{document}
\section{Introduction}

The motions and movements of HII regions are largely not understood. Spectral data taken in the past have suggested that HII regions may have a simple spin about an axis, but followup observations seem to shoo away this theory. It may be suggested that there is a turbulance that drives the discrepancies between the more simple model and the observations. To do this, we observe the HII region [put the important HII region here or data set]. Notably this/these HII regions have an unusual shape that [is important because why?]. Bipolar HII regions such as these are HII regions that are shaped so they have two lobes, typically thought to be shaped by outflows from energetic young stars. The goal of this work is to analyze the link between the kinematics of the hotter HII region and the motion of the surrounding ISM. This will be done by comparing spectral data of radio recombination lines from the HII region and the infrared spectral data of surrounding neutral gas. 

When an HII region is observed in the 4-10GHz regime, the primary emissions noticed are free-free (FF) and radio recombination line (RRL) emission. FF emission is driven by electrons passing by ions, having some of their kinetic energy converted into a photon. This creates a continuum of radio emission. RRL emission is caused by atoms with a high principal quantum number descending in energy. This is similar to the process of the creation of say, the H$\alpha$ emission line, which goes from $n=3$ to $n=2$. In the case of our RRL, this would instead be something like from $n=100$ to $n=99$, so the emitted photon has less energy, putting it in the radio frequencies.

To process the data of the radio recombination lines, we will be making use of multifrequency synthesis. In this process, we ???. This is done to maintain the resolution of our data even after combining several frequency ranges together, so we may have a high sensitivity as well.

\section{Psuedo-Code}
In order to test the validity of the discrepencies seen being driven by turbulence, we simulate an HII region. First, two 3d matrices with the same resolution as the desired data cube images are created. The first one models density via equation \ref{eq:density_flux}, having the distribution of densities be generated by the python package \code{turbustat}, using a power-law of 11/3. The second one represents the velocity and is generated with an rms of the mach number of the gas and $c_s$, with a power-law of 5/3. The densities and velocities are then used to make a spherical HII region, also given a radius, temperature, and the distance from the observer. From there, this HII region is given to a simulator that "observes" the HII region over several frequencies, given a number of channels, a pixel size in units of angle, the size of each channel, and the frequency of a radio recombination line. The simulation calculates the emission measure for each point in the 3d matrix of the HII region. Then, using the velocities for the peak emission frequency for each pixel, the RRL brightness is calculated for each channel in our data cube. This cube is saved so that it can have noise applied and be convolved with a beam to simulate a radio observation.

\begin{equation}
  \rho
  \label{eq:density_flux}
\end{equation}

\section{Abstract}
\section{Introduction}
\subsection{HII Regions}
HII regions are regions of space which contain a young star, surrounded by hydrogen.

Primary emissions observed in radio regime.
\subsection{Turbulence}
Turbulence loosely follows rules that allow us to predict how it will act, this paragraph discusses how.
\section{Methods}
\subsection{Equations for Radiative Transfer}
This section will discuss the equations used, where from, and the approximations made when talking about how we tested the HII regions.
\subsection{Processes Used}
In this section, we outline the processes used in creating and processing simulations of HII regions.

In this paragraph, we outline how the turbulence is generated and what kind of parameters we use in order to imitate reality as closely as possible.

We then go on to discuss how we implemented each of the equations of radiative transfer into the code.

We then discuss the convolution of the radio images so that the simulations imitate radio images.

Then the process of talking about the processing of the images into their velocity maps is done.

Finally, the use of PowerSpectrum is explained for getting power-law.
\section{Results}
\subsection{First Glance, Visually}
Talk about the emergence of bipolarity, how it could plausibly be caused by the density difference from the turbulence.

Show the first moment map with standard parameters, noting the emergent shapes with different beamwidths.
\subsection{Behavior wrt Mach \& Driving Number}
Explain the variation in the VLSR statistics, including the power-law.
\subsection{Different Stars as Parameters}
Experimental little section, where we create a table that notes difference in stats between various types of stars.
\section{Discussion}
\subsection{Comparing to Real Data}
Discussion of the data processed for this project, including how the simulation fits into the data provided by it.
\subsection{Next Steps}
Whatever I haven't covered in this paper, I am sure that there is more that I can talk about here
\section{Conclusion}

\end{document}
